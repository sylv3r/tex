\documentclass[8pt,french]{base}

\usepackage{cgv}

\usepackage{multicol}
\setlength{\columnsep}{0,8cm}

\def \companyname {SCYON LABS}

\begin{document}

\maintitle{CONDITIONS GÉNÉRALES DE VENTE ET DE SERVICE}

\begin{multicols*}{3}

\section*{Préambule}

\companyname\ est une société à responsabilité limitée au capital de 2~000~Euros, dont le siège social est situé 101 rue de Sèvres - 75006 PARIS, immatriculée au registre du commerce et des sociétés de Paris sous le numéro  531 533 396.

\section{Objet du contrat}

Les présentes Conditions Générales (ci-après les «~CG~») ont pour objet de fixer les conditions que le Prestataire et le Client s’engagent à respecter pour les prestations prévues dans le Bon de Commande joint aux présentes CG. Seul le présent contrat fera foi au titre des conventions acceptées entre les parties. Aucun autre document sauf stipulation expresse ne sera pris en considération.

\section{Documents contractuels}

Le Contrat se compose des documents contractuels suivants~:
\begin{itemize}
  \item Le Bon de Commande
  \item Les Conditions Générales
\end{itemize}
Ce classement par ordre de priorité décroissante prévaut en cas de contradiction ou de difficulté d’interprétation.

\section{Nature des obligations du Prestataire}

La Prestation se traduit par la mise à disposition des ressources demandées par le Client et/ou la fourniture des éléments à réaliser dans le cadre de la prestation, déterminées dans le Bon de Commande joint aux présentes CG. \\
Les obligations du Prestataire consistent à faire bénéficier le Client de sa compétence technique en mettant en place des moyens techniques spécifiques affectés à la réalisation de la Prestation. \\
Le Prestataire s’engage à effectuer cette mission avec les moyens et/ou l’effectif nécessaire à la bonne exécution de la Prestation.

\section{Conditions d’exécution de la Prestation}

\subsection{Maîtrise d’ouvrage}

Le Client assure seul la maîtrise d’ouvrage de l’ensemble du Projet et, à ce titre, désigne un chef de projet investi d’un pouvoir de décision et possédant les compétences techniques lui permettant de prendre toute décision à l’égard des propositions présentées par le Prestataire. Son identité devra figurer explicitement dans le Bon de Commande.

Le Client s’engage à collaborer avec le Prestataire pour lui permettre d’exécuter la Prestation, notamment~:
\begin{itemize}
  \item En fournissant au Prestataire les informations et éléments indispensables à la bonne compréhension du contexte et des enjeux~;
  \item En définissant ses besoins le plus précisément possible~;
  \item En mettant à la disposition du Prestataire les moyens nécessaires~;
  \item En assurant les liaisons nécessaires entre ses propres services et le Prestataire~;
  \item En assurant la prise de connaissance par le Prestataire des contextes techniques et fonctionnels spécifiques utiles à la réalisation de la prestation~;
  \item En approuvant ou formulant toute observation par écrit, concernant la réalisation des prestations commandées et les documents qui lui sont remis par le Prestataire.
\end{itemize}

% cut ici des lieu, durée et absence de presta, à rajouter avec une condition pour etre générique !

\section{Modalités de paiement}

\subsection{Facturation}

Le Prestataire facturera le Client :
\begin{itemize}
  \item dans le cadre d’une régie, à chaque fin de mois en application du prix journalier indiqué dans le Bon de Commande sur la base de la feuille de temps visée par le Client~;
  \item dans le cadre d’un forfait, comme défini dans le Bon de Commande ou, à défaut, à la signature de la Commande par un tiers d’acompte, puis du solde à la réalisation de l’ensemble de la prestation définie dans le Bon de Commande.
\end{itemize}

\subsection{Frais de déplacement}

Si, à la demande du Client, le Prestataire était amené à se déplacer en dehors de la région parisienne, ses entiers frais, notamment de déplacement et d’hébergement, seraient facturés en sus sur présentation des justificatifs, avec une majoration de quinze pourcents (15\%) pour peines et soins. \\
Le Prestataire s’engage, dans ce cas, à prévoir d’avance ces frais et présenter une estimation de ceux-ci au Client avant de les engager.

\subsection{Règlement}

Toutes les factures du Prestataire sont payables dans les délais et selon les modalités fixées dans le Bon de Commande.

\subsection{Défaut de paiement}

De convention expresse, le défaut de paiement à l’échéance, après mise en demeure d’exécuter adressée par courrier recommandé avec avis de réception et non satisfaite dans un délai de quinze (15) jours calendaires, autorise le Prestataire, outre son droit à dommages et intérêts~:
\begin{itemize}
  \item à suspendre de plein droit et sans formalité la Prestation~;
  \item à exiger immédiatement toutes les sommes dues~;
  \item à facturer, à compter de l’échéance, un intérêt de retard au taux légal majoré de 1,5 points~;
  \item à suspendre sans formalité tout service hébergé par le Prestataire (serveur, site internet, boite aux lettres électronique, ...) au bénéfice du Client.
\end{itemize}

\subsection{Révision annuelle}

Dans le cadre d'une régie, le prix des Prestations sera automatiquement révisé tous les ans au 1er Janvier de la nouvelle année. \\
La révision sera basée sur la formule suivante~: P = Po x (S/So)
Sachant que~:
\begin{itemize}
  \item P = prix révisé
  \item Po = prix initial de la prestation
  \item So = valeur du dernier indice SYNTEC publié à la date où le présent contrat a été établi
  \item S = valeur du dernier indice SYNTEC publié à la date de révision
\end{itemize}
{\itshape L’indice SYNTEC en Octobre 2017 (So) était de 263,8} \par

Si l'indice ci-dessus venait à disparaître, les Parties lui substitueront
un indice de remplacement. A défaut, un nouvel indice sera choisi par le Tribunal de commerce de PARIS statuant à la requête de la partie la plus diligente.

\subsection{Interventions exceptionnelles}

A la demande du Client, des interventions exceptionnelles pourront
avoir lieu en dehors des horaires ouvrés habituels (Pour ce besoin, sont définis comme «~horaires ouvrés habituels~» du Lundi au Vendredi de
9h00 à 13h00 et de 14h00 à 18h00). \\
Ces interventions devront être validées au préalable par un échange écrit (papier ou électronique) entre les Parties. \par
Elles seront facturées au taux horaire majoré en tenant compte des conditions suivantes~:
\begin{itemize}
  \item Samedi~:~150\%
  \item Dimanche~:~200\%
  \item Jour Férié~:~200\%
  \item Nuit semaine~(21h-7h)~:~150\%
\end{itemize}

\section{Entrée en vigueur}

Les Conditions Générales prennent effet à compter de leur date de signature et expireront trois (3) ans après leur signature par les Parties. \\
À l’expiration de cette durée initiale et à défaut de dénonciation par une des deux Parties dans un délai d’un (1) mois avant le terme contractuel, les Conditions Générales seront tacitement prolongées pour une durée indéterminée. En conséquence, chaque Partie pourra résilier les Conditions Générales à tout moment par lettre recommandée avec demande d’avis de réception moyennant le respect d’un préavis de trois (3) mois. \\
Le Contrat prend effet à la date indiquée dans le Bon de Commande.

\section{Résiliation}

En cas de manquement grave par une Partie à l’une quelconque des obligations mises à sa charge par le Contrat, celui-ci pourra être résilié de plein droit par l’autre Partie, sans préjudice de toute autre action, quinze (15) jours après que soit restée infructueuse une mise en demeure de faire cesser le dit manquement adressée à la Partie défaillante par lettre recommandée avec accusé de réception. \\
En cas de résiliation, la Prestation sera payée par le Client au Prestataire pour sa partie déjà réalisée. Sous réserve de parfait encaissement des sommes qui lui seraient dues, le Prestataire livrera au Client le résultat de la Prestation pour sa partie réalisée et lui restituera l’ensemble des documents et des éléments qui auront été mis à sa disposition conformément au Contrat.

\section{Responsabilité}

La responsabilité du Prestataire ne pourra être recherchée par le Client sauf sur la base d’une faute prouvée. \\
La responsabilité du Prestataire ne pourra être recherchée dans le cas où les dommages invoqués par le Client seraient consécutifs à une inexécution, même partielle, des obligations incombant au Client et plus généralement pour tout dommage indirect tel que préjudice commercial, perte d’exploitation, manque à gagner, perte de clientèle, conséquence d’action dirigée contre le Client par un tiers. \\
Au cas où la responsabilité du Prestataire serait retenue, les Parties conviennent expressément que toutes sommes confondues, l’indemnité versée au Client en réparation de son préjudice sera limitée aux sommes effectivement perçues par le Prestataire au titre de la prestation en cause, et ce, quel que soit le fondement juridique de la réclamation et de la procédure employée pour la faire aboutir. \\
Dans ce cadre, le Prestataire certifie qu’il est assuré auprès d’une compagnie d’assurances notoirement solvable pour toutes les responsabilités qu’il pourra encourir au titre du présent contrat. Il doit ainsi garantir sa responsabilité civile professionnelle et la responsabilité du fait des dommages de toute nature que ses intervenants pourraient causer au Client, à ses préposés ou à des tiers dans l’exécution du contrat. Il devra justifier vis-à-vis du Client dès la commande, et à toute demande de celui-ci, de la souscription des polices couvrant les risques en cause, ainsi que du paiement régulier des primes.

\section{Conditions indépendantes de la volonté des parties}

Les Parties ne pourront être tenues responsables pour un manquement à l’une des obligations mises à leur charge par le Contrat, qui résulterait de circonstances indépendantes de leur volonté, telles que, mais sans limitation, grève ou conflit du travail, guerre ou autre acte de violence, cas fortuit, défaillance imputable à la force majeure, sous réserve toutefois que la Partie invoquant de telles circonstances notifie leur existence dès que possible, qu’elle fasse de son mieux pour en limiter les conséquences, et enfin qu’elle reprenne l’exécution du Contrat immédiatement après que ces circonstances auront disparu. \\
Dans la mesure où de telles circonstances se poursuivraient pendant une durée supérieure à un (1) mois, l’une quelconque des Parties pourra résilier le Contrat, sans dommages ni intérêts vis-à-vis de l’autre Partie, sur notification par lettre recommandée avec accusé de réception. La résiliation prendra effet à réception de cette notification.

\section{Confidentialité}

Les Parties s’engagent à mettre en œuvre les moyens appropriés pour garder le secret le plus absolu sur les informations et documents communiqués dans le cadre du Contrat et auxquels elles auraient eu accès à l’occasion de sa négociation ou de son exécution. \\
En particulier, les Parties s’engagent à garder le secret le plus absolu sur les méthodes utilisées et dont elles auraient eu connaissance dans le cadre du Contrat. \\
Toutefois, les Parties ne seront pas responsables de la divulgation d’informations si elles sont déjà dans le domaine public ou si elles ont été obtenues par d’autres sources et en l’absence de faute de la Partie qui les a reçues. \\
Cet engagement se poursuivra pendant un (1) an après la cessation du Contrat. \\
Nonobstant ce qui précède, le Prestataire pourra, dans le cadre de ses activités, citer le nom du Client à titre de référence.

\section{Propriété Intellectuelle}

Dans le cadre d'une régie, il s'entend que la propriété intellectuelle de tout développement, conception et réalisation de quelque nature que ce soit effectués par le Prestataire pour le compte du Client est intégralement reversé, et ce sans condition, au Client. \\
Dans le cadre d'un forfait, tout ou partie de la propriété intellectuelle du ou des livrable(s) concerné(s) par le Bon de Commande peut être reversée au Client, selon conditions définies dans le Bon de Commande. Il s'entend à défaut que le Client en a l'usage exclusif, l'exploitation et le droit de modification, mais ne peut en aucun cas le ou les diffuser, reproduire ou revendre, notamment pour un bénéfice personnel au détriment du Prestataire.

\section{Non sollicitation du personnel}

Le Client s'interdit d'engager, ou de faire travailler d'aucune manière, tout collaborateur présent ou futur de \companyname. \\
La présente clause vaudra, quelle que soit la fonction du collaborateur en cause, et même au cas où la sollicitation serait à l'initiative dudit collaborateur. La présente clause déroulera ses effets pendant toute l'exécution du présent contrat, et ce pendant douze (12) mois à compter de sa terminaison, sauf négociation expresse entre les Parties.

\section{Intégralité}

Le Contrat exprime l’intégralité des droits et obligations des Parties et annule et remplace toutes les conventions orales ou écrites qui auraient pu être passées antérieurement entre les Parties concernant la Prestation. \\
Le fait qu’une disposition ou une clause de ce contrat s’avère illégale ou inexécutoire n’aura pas pour effet d’annuler le contrat, seule la disposition ou la clause litigieuse sera considérée comme nulle et non avenue.

\section{Compétence et droit applicable}

Le Contrat est soumis en toutes ses dispositions et conséquences à la loi française. \\
Toute réclamation du Client à l’encontre du Prestataire devra être formulée par le Client au plus tard huit (8) jours à compter du fait générateur sous peine de déchéance. \\
Tout différend se rapportant à l’interprétation, l’exécution, la rupture du contrat ou toute difficulté à défaut de règlement amiable seras soumis au tribunal de commerce de PARIS ou à la juridiction compétente de PARIS. \\
Les Parties conviennent dès à présent par accord exprès d’attribuer compétence exclusive aux tribunaux de PARIS même en cas d’appel en garantie et pour les procédures conservatoires, en référé ou par requête.

\vspace{3em}

\noindent
Fait à Paris, le 17 Janvier 2018\\
(En deux exemplaires originaux)\\
\par
\noindent
\textbf{OROSOUND} – Eric BENHAIM \par
{\small "Lu et Approuvé" \\
(date et cachet de l'entreprise)}

\end{multicols*}

\end{document}
